\documentclass{article}

\usepackage{amsfonts}

\title{Tema 2 - Sucesiones de números reales}
\author{Raúl Gilabert Gámez}

\begin{document}
	\begin{titlepage}
		\maketitle
	\end{titlepage}

	\newpage

	\section{Introducción a las secuencias}

	Una sucesión es una aplicación de un conjunto de números naturales a números reales, de tal manera que es así:
	$a:D \longrightarrow\mathbb{R}$;
	$D\subset \mathbb{N}$

	\bigskip
	
	Notación:

	1.- \newline
	$Im_a \equiv (a_n)_{n \in D \land n \geq 0}$

	\bigskip
	2.- \newline
	$a_1$ es el primer elemento de la sucesión. \newline
	$a_2$ es el segundo elemento de la sucesión. \newline
	$a_n$ es el n-ésimo elemento de la sucesión. \newline

	Formas más frecuentes de dar una sucesión: \newline

	1.- \newline
	Dando los primeros elementos $\rightarrow$ 0, 2, 4, 6, 8 \newline

	2.- \newline
	Dando el término n-ésimo $\rightarrow$ $\forall_{n \geq 0} (a_n = 2n) $ \newline

	3.- \newline
	Por referencia $ \rightarrow \left\{ \begin{array}{rcl} a_0 & = & 0 \\ a_n & = & a_{n-1} + 2, n \geq 0 \end{array}\right.$

	\bigskip

	\section{Límites de secuencias}

	\end{document}